\section{Elections Policy}
\index{Election!Policy}
\begin{longenum}[ label*=\thesection.\arabic*., align=left]
 \item \textbf{Appeals Review Commission}
 \begin{longenum}[label*=\arabic*., align=left]
  \item Appeals Review Commission (ARC) shall be elected annually at the Society's November Council meeting.
  \item Appeals Review Commission (ARC) shall have a ranked waiting list
  \begin{longenum}[label*=\arabic*., align=left]
  	\item  During the November council meeting, a waiting list of no less than one member shall be created.
  	\item Should a member of the Appeals Review Commission no longer be able to fulfil their duty, the highest ranked member from the waiting list shall take their place.
  \end{longenum}
  \item Any member who joins the Appeals Review Commission or Appeals Review Commission waiting list must state their intention to not graduate within four (4) months.
 
 \end{longenum}
 \item \textbf{Presidential Nominations}
 \begin{longenum}[label*=\arabic*., align=left]
  \item  Nominations for the position of President shall be open from the first Monday following January 2nd for ten (10) consecutive business days, closing on the tenth day at the close of business hours. 
  \item  Nomination forms must be obtained from the Society office during the Society's business hours. Nominations shall bear the signatures of two full, associate, or leave-of-absence members of the Society as well as that of  the nominee, and shall be submitted to the Chief Returning Officer during the Society's business hours.
  \item Appeals regarding a decision on the validity of a nomination shall be made in writing to the Chief Returning Officer no later than one business day after the announcement nomination. The Chief Returning Officer shall render a decision before the beginning of the campaign period.
  \item The call for nominations shall be advertised by the CRO one week prior to the opening of nominations.
  \item In the event that no nominations for the position of President be received by the Chief Returning Officer by the close of nominations, nominations shall be reopened the first Monday in February and close after ten (10) consecutive business days, at the close of business hours.
  \item In the event that no nominations be received by the end of the second nomination period, the Council shall nominate an eligible member of the Society at the Council meeting following the close of nominations. If Council fails to do so, then the Executive shall nominate an eligible member of the Society as President, subject to ratification at the next Council meeting. If ratification is not forthcoming, Council must appoint a President at that meeting. If Council fails to do so, then the new President shall be selected by the Executive from the Vice-Presidents.
  \item The new President shall take the following Oath at the Council meeting following their election: ``I,[name], do hereby swear to faithfully fulfill my duties as President and to uphold the Constitution and Bylaws of the Society of Graduate Students of the University of Western Ontario." The Oath shall be administered by the Speaker.
  \item In the event that the elected President is a departmental or part-time representative, Speaker, or another member of the Executive, that person shall not hold the position of departmental representative, Speaker or other Executive member while being President.
  \item Actual transfer of signing authority and responsibilities of the Office of the President shall take place on the first of May, at which time the retiring President shall formally hand over office to the incoming president. In the event that there is a president to take the office, otherwise it will happen as soon as practicable
  
 \end{longenum}
 \item \textbf{Campaigning}
  \begin{longenum}[label*=\arabic*., align=left]
 
 \item Campaigning is defined as any action by a candidate or campaign manager, or any action undertaken at the behest of the candidate or campaign manager, that is intended to influence any voter to cast their ballot on behalf of the candidate in question.
\item The campaign period shall begin two business days after nominations close and shall continue up to and including the day preceding the balloting. No campaigning may be undertaken other than during the designated campaign period.
\item Prior to the beginning of the campaign period, all parties and their campaign managers shall attend a meeting wherein they will be briefed on the terms of the election by the Chief Returning Officer and Deputy Chief Returning Officer.
\item Campaigning must cease by midnight (11:59 pm local time) before the day of balloting. All advertising must be removed by this deadline. 
    \index{Election!Campaign!End of Period}
    \index{Referendum!Campaign!End of Period}
\item At that meeting an agreement will be signed by all parties wherein they acknowledge being bound to conduct their campaign in accordance with the Society's Constitution and Bylaws, and define any terms of the election not covered by these Bylaws.
\item All  forms of media may be used during  the campaign period. No media coverage is allowed on  the day of balloting.
\item  Society resources or materials may not be used in the preparation of campaign material, except with the agreement of all candidates and the Chief Returning Officer.
\item Society space may not be used in the dissemination of campaign material, except with the agreement of all candidates and the Chief Returning Officer.
 \end{longenum}
 \item 	\textbf{Advertisements	of	a	Presidential	Election	or	Referendum	by	the	Society }
  \begin{longenum}[label*=\arabic*., align=left]
  \item Advertisements shall be placed:
 \begin{longenum}[label*=\arabic*., align=left]
\item on campus bulletin boards one month prior to the final voting date;
\item on the Society's Web page and in the monthly newsletter;
\item Any other place as the CRO considers appropriate and effective within budgetary 
constraints.
 \end{longenum}
\item Advertisements shall consist of date(s) and place(s) or voting, the candidates' names or referendum 
statement(s) and the name of the Society. 
 \end{longenum}
  
 \item \textbf{Scrutineer}
 \index{Election!Scrutineer}
 \begin{longenum}[label*=\arabic*., align=left]
\item Each candidate/referendum group shall be allowed one scrutineer to be present when the final results of the election/referendum, as recorded by the on-line polling station, are revealed.
 \end{longenum}
 \item 	\textbf{Campaign Impropriety and Appeals Policy}	
 \begin{longenum}[label*=\arabic*., align=left]
\item No candidate/referendum group or individual member may, using their own initiative and 
discretion, attempt to enforce the rules for elections.
\item Campaign Impropriety is defined as any action undertaken by a candidate or their representative 
during the campaign and polling period that can be shown to be violations of any part of the Society's 
Constitution, Bylaws, or the agreement between the parties reached at the All Candidates Meeting.
\item To find that an act of campaign impropriety has occurred, the CRO must be satisfied that the result of 
the Deputy CRO's investigation shows, on the balance of probabilities, the indicted party did commit the 
offence with which they have been charged.
\item The investigation
\begin{longenum}[label*=\arabic*., align=left]
\item All charges of campaign impropriety shall be submitted in writing to the Deputy Chief Returning Officer, (Deputy CRO) at the Society office or by email to the official CRO/DCRO email address;
\item Within one business day of the charge having been submitted the Deputy CRO will begin an impartial investigation and will also notify the CRO of the pending investigation; 
\item Barring exceptional circumstances, the Deputy CRO shall submit the result of their investigation to the CRO no more than one business day after the start of investigation.
\end{longenum}
\item Should a candidate/referendum group receive a request from the Deputy CRO to provide factual information which wholly pertains to an investigation of campaign impropriety, the candidate/ referendum group must respond within one half of the investigation time or twenty-four (24) hours, whichever is greater. Failure to respond within the time frame will result in a report of nonresponse in the Deputy CRO's written report.
\item The Chief Returning Officer shall decide on any charge or campaign impropriety, no more than four (4) business days following the complaint.
\item Sanctions
	\begin{longenum}[label*=\arabic*., align=left]
\item It is at the CRO's discretion to determine whether a sanction shall be one of the following three types: Warning, Minor, or Major. The CRO, DCRO, candidates, and their campaign managers shall determine the parameters of what constitutes campaign impropriety at the All Candidates Meeting (see Bylaw 2.4.6.2.).
\item Warning – the CRO shall issue a warning to the candidate via email regarding their campaign impropriety.
\item Minor Sanctions – The CRO may use the office's authority to rectify the situation in the event of a minor sanction (see Bylaw 2.4.10.).
\item Major Sanctions – Violations of the following nature will result in automatic disqualification of the candidate:
	\begin{longenum}[label*=\arabic*., align=left]
\item Failure of the candidate and their campaign manager to attend the All Candidates' Meeting with the Chief Returning Officer and Deputy Chief Returning Officer;
\item Tampering with other candidates' signs so as to cause their being defaced or removed;
\item Spending 125\% or more of the maximum spending limit;
\item Violations of Canadian and Ontario law may result in disqualification at the CRO's discretion.
\end{longenum}
\end{longenum}
 \end{longenum}
 
 \item \textbf{Appeals}
 \index{Election!Appeals}
 \index{Election!Elections Appeals Commission}
 \begin{longenum}[label*=\arabic*., align=left]
 \item The Deputy CRO shall investigate the appeal and present the results of that investigation to the CRO within  two (2) business days. The CRO will  take appropriate action in response  to  the investigation results within two (2) business days.
\item  When  the  Chief  Returning  Officer  has  rendered  a  decision  on  the  submission,  a  further  written appeal may be taken to the Appeals Review Commission within two (2) business days of the announcement of the decision.
\item The onus is on the appellant to appeal the decision of the CRO in writing to necessitate the activation of the Appeals Review Commission.
  \end{longenum}
  \item \textbf{Campaign Expenses and Subsidy Policy}	
  \begin{longenum}[label*=\arabic*., align=left]
  \item The spending limit will be \$150 unless changed at the February Council meeting by a motion which requires a simple majority to pass.
\item \label{Presidential Subsidy}The Society shall provide full subsidies for campaign expenses incurred by each referendum group or presidential candidate that receive at least 10\% of the total unspoiled votes cast in the election.  
			\begin{longenum}[label*=\arabic*., align=left]
			\item  Acclaimed  presidential  candidates  are  assumed  to  have  passed  the  10\%  threshold  in section \ref{Presidential Subsidy}.
			\end{longenum}
\item All candidates shall submit to the Chief Returning Officer documentation of all expenditures by the election day.
 \end{longenum}
\end{longenum}
