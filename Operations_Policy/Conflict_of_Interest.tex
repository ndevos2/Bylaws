\section{Conflict of Interest}\label{policy: coi}

\subsection{Governing Bylaws}
\begin{enumerate} [align=left]
\item The Society of Graduate Students bylaws section 2.11 (Conflict of Interest) shall be used to
determine whether or not a Conflict of Interest exists.
\end{enumerate}

\subsection{Procedure for Resolution}
\begin{enumerate} [align=left]
\item An individual may recognize one's own conflict and follow the procedure outlined in Bylaw 2.11.2 to resolve a conflict. At any point in the following procedural outlines and policies, a person may acknowledge their own conflict and seek steps to resolve it privately, in accordance with the Bylaws. This should always be encouraged in the interests of saving the Society unnecessary time and energy expended resolving its internal difficulties.
\item An individual must first inform the person perceived to be in a Conflict of Interest of their conflict, and must do so clearly, substantively and in writing (electronic media are acceptable). Vague allegations, generalized concerns and innuendo are not substantive claims.
\item Failing to resolve the matter privately and discretely may result in an escalation to the conflicted person’s supervisor, in whatever capacity they may be supervised: a committee chairperson, a committee's official liaison, the Society's President, or the Society's Speaker (in order of preference, where applicable). Third parties external to the Society are not considered part of any supervisory chain, and may not be employed.
\item Failing resolution through a supervisory intervention, the concerned party may do one of two things, depending upon the time sensitivity of the issue (to be determined by the Speaker):
\begin{enumerate} [label*=\arabic*., align=left]
\item If the issue is not time-sensitive, the individual may draft a motion for Council and seek resolution in that forum.
\item If the issue is of a time-sensitive nature, the individual may seek an ad-hoc tribunal proceeding through the Society’s Speaker (see 3.1.3.3, Tribunal Proceedings).
\item The only bodies capable of imposing a decision on a person perceived to be in a Conflict of Interest are the Society’s Council (2.2.3.1) and the ad-hoc tribunal formed by the Speaker. All other methods of resolution must come in the form of recommendations to the allegedly conflicted individual, and to which, all parties involved must agree.
\item  At each stage of attempted resolution, five (5) business days must be allowed to
respond.
\end{enumerate}
\end{enumerate}

\subsection{Tribunal Proceedings}
\begin{enumerate} [align=left]
\item  In the event a perceived Conflict of Interest remains unresolved and is of a time-sensitive nature that cannot wait until a Council meeting, the Speaker may call an ad-hoc tribunal to impose a judgement upon the conflict situation in question.
\item The ad-hoc tribunal shall be called if the complaint is received in writing (electronic media are acceptable) to the Society’s office and addressed to the Speaker.
\item A tribunal shall consist of five (5) Society members in good standing: the Speaker (non-voting), the Ombudsperson (non-voting), and three other Society members representing three different faculties. These members shall be selected by the Speaker, and should themselves be free of any reasonable apprehension of bias toward both the complainant and defendant.
\item The tribunal shall assemble within seven (7) business days, barring unforeseen circumstances, and shall offer a ruling based on the most complete testimony of the individuals involved. Follow up queries are permitted for clarification, and at all points the Speaker shall provide guidance on the interpretation of the Society’s Bylaws to assure that the minimum standards for a Conflict of Interest are met.
\item Tribunal decisions must be passed by a majority vote.
\end{enumerate}	

\subsection{Possible Resolutions}
\begin{enumerate} [align=left]
\item Any person, supervisor, Council, or ad-hoc tribunal may determine that no conflict exists, and thus dismiss the charge. Reasons should be documented as best as possible, with resolutions ready to be provided in the event the matter is escalated.
\item Any person or supervisor (as described in 3.1.2.2 and 3.1.2.3) may determine that a conflict does exist, and thus may recommend the following, in order of desirability:
\begin{enumerate} [label*=\arabic*., align=left]
\item The conflicted individual be asked to cease participating in the situation generating the conflict, either by noted abstention during voting, by leaving the room during a committee meeting in which the issue arises, or some other similarly appropriate and generally benign measure of resolution.
\item The conflicted individual may be asked to step down from their position of authority or influence in the Society, and from which the conflict is derived.
\item The conflicted individual may bring the matter forward to the Speaker to be resolved by Council or an ad-hoc tribunal (as described in 3.1.2.4).
\end{enumerate}
\item Council or an ad-hoc tribunal may determine that a conflict does exist, and thus may recommend or impose the following sanctions, in order of desirability:
\begin{enumerate} [label*=\arabic*., align=left]
\item The conflicted individual be asked to cease participating in the situation generating the conflict, either by noted abstention during voting, by leaving the room during a committee meeting in which the issue arises, or some other similarly appropriate measure.
\item The conflicted individual may be asked to step down from their position of authority or influence in the Society, and from which the conflict is derived.
\item The conflicted individual may be censured and face no further disciplinary actions. 
\item The conflicted individual may be censured and face additional disciplinary measures as outlined in Bylaw 2.19 (Disciplinary Measures), or more severe measures such as:
\begin{enumerate} [label*=\arabic*., align=left]
\item A ban from committee proceedings, or other specific activities, or
\item A ban from the Society not exceeding twelve (12) months.
\end{enumerate}
\item In all instances where a judgement is being proffered, rather than a mere recommendation, the matter shall be presented to Council for formalization and documentation.
\begin{enumerate} [label*=\arabic*., align=left]
\item Recommendations, provided they achieve a resolution, need not be formalized in any capacity as these should be interpreted as successful private mediation. Such resolutions (3.1.4.2.1 or 3.1.4.2.2) should never be brought to the formal attention of Council or recorded in its minutes.
\end{enumerate}
\item In all instances where measures are taken against a conflicted individual, said measures shall be rationalized in writing that the measure does not exceed the charge against the conflicted person.
\end{enumerate}
\end{enumerate}