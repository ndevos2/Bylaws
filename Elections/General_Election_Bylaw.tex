\subsection{Responsibility}
\index{Election!Responsibilities of the CRO}
\begin{longenum}[ label*=\thesubsection.\arabic*., align=left]
	\item The Chief Returning Officer (CRO) in conjunction with the Deputy Chief Returning  officer, as per bylaw 2.3, shall be responsible for all aspects of an election or referendum, including but not limited to:
    \begin{longenum}[ label*=\arabic*., align=left]
		\item call for nominations; \index{Election!Call of Nomination}
        \item announcement of candidates;
        \item establishment of an on-line balloting system and the proper functioning of that system; \index{Election!Balloting System}
        \item announcement of the dates of balloting; \index{Election!Date of Election}
        \item ruling on the validity of any election or referendum;
        \item and announcement of official results to candidates, referenda groups, Council, and the media. \index{Election!Anouncement of Result}
	\end{longenum}
    
\end{longenum}

\subsection{Appeals Review Commission}
\index{Appeals Review Commission!Mandate}
\begin{longenum}[ label*=\thesubsection.\arabic*., align=left]
	\item The Appeals Review Commission shall rule on appeals relating to presidential elections and referenda.
    \item The Appeals Review Commission must abide by the following when making rulings: 
    \begin{longenum}[ label*=\arabic*., align=left]
    	\index{Appeals Review Commission!Rules of Procedure}
		\item The onus is on the appellant to demonstrate the charge of impropriety was not correct;
        \item The appellant has the right, but not the obligation, to appear before the Appeals Review Commission at a hearing to present argument or evidence; 
        \item The Appeals Review Commission has the right to initiate and hold an Appeals Review Commission Hearing for an appeal under consideration; 
        \item An Appeals Review Commission Hearing shall request the appearance of individuals relevant to matters of an appeal:
        \begin{longenum}[ label*=\arabic*., align=left]
        	\index{Appeals Review Commission!Summons}
			\item The Appeals Review Commission has the right to demand the presence of the Chief Returning Officer and/or the Deputy Chief Returning Officer at an Election Appeal Hearing; 
            \item The Appeals Review Commission may request, but not demand, the presence of any person not mentioned in Bylaw 2.4.2.2.4.1 at an Election Appeal Hearing;
            \item The procedure and handling of evidence at an Appeals Review Commission Hearing is governed by these bylaws and policies as well as Roberts Rules of Order.
            
		\end{longenum}

	\end{longenum}
    
    \item Members shall be elected to the Appeals Review Commission at the Society's November Council Meeting.
    \index{Election!Appeals Review Commission}
    \index{Appeals Review Commission!Election}
    \item Quorum for a meeting of the Appeals Review Commission shall be four (4).
    \index{Appeals Review Commission!Quorum}
    \index{Quorum!Appeals Review Commission}
\end{longenum}

\subsection{Referendum-Specific Rules}
\index{Referendum}
\begin{longenum}[ label*=\thesubsection.\arabic*., align=left]
	\item Need for a referendum, as well as the wording of the referendum statement, shall be determined and approved by either a General Meeting of the Society or Council. 
    \item A campaign spending limit shall be established by Council for each referendum group.
    \index{Referendum!Spending Limit}
    \index{Election!Spending Limit}
    \item The Chief Returning Officer shall present the decision regarding validity of the referendum to Council for procedural ratification: 
    \begin{longenum}[ label*=\arabic*., align=left]
		\item if valid, the results of the voting shall be binding on the Society;
		\index{Referendum!Binding on Society}
        \item if invalid, the referendum will be re-run at a date determined by Council.
        \index{Referendum!Invalid}
	\end{longenum}
    
\end{longenum}

\subsection{Graduate Student Representative to the Board of Governor and Graduate Student Representative to the Senate Election Rules}
\index{Board of Governor!Election}
\index{Senate!Election}
\index{Election!Senate}
\index{Election!Board of Governor}
\begin{longenum}[ label*=\thesubsection.\arabic*., align=left]
 
\item Timeline for the elections relating to the Board of Governor and University Senate
shall be set by the University Secretariat
\item  All Graduate Students (as defined by the School of Graduate and Postdoctoral Studies) are eligible to vote in this election
\end{longenum}


\subsection{Presidential Election-specific Rules}
\index{President, The!Election}
\index{Election!President}
\begin{longenum}[ label*=\thesubsection.\arabic*., align=left]
	\item Timeline for the Presidential Election shall be set by the University Secretariat.
    \item All full, associate, and leave-of-absence members shall be eligible to vote. \index{Election!Suffrage}
\end{longenum}

\subsection{Campaigning}
\index{Election!Campaign}
\index{Referendum!Campaign}
\begin{longenum}[ label*=\thesubsection.\arabic*., align=left]
    \item Prior to the beginning of the campaign period, all parties and their campaign
managers shall attend a mandatory all candidates meeting wherein they will be briefed on the terms of the election by the CRO and DCRO.
    \index{Election!Campaign!All Candidates Meeting}
    \index{Deputy Chief Returning Officer!All Candidates M eeting}
    \index{Referendum!All Candidates Meeting}
    \index{Chief Returning Officer!All Candidates Meeting}
	\item Prior to the beginning of the campaign period, all parties and their campaign managers shall attend a mandatory all candidates meeting wherein they will be briefed on the terms of the election by the CRO and DCRO.
\end{longenum}


\subsection{Transition}
	\index{President!Transition}
	\index{President!President-Elect}
	\index{President-Elect}

\begin{longenum}[ label*=\thesubsection.\arabic*., align=left]
	\item For the month prior to taking office, the President-Elect shall:
	 \begin{longenum}[ label*=\thesubsection.\arabic*., align=left]
	\item Be expected to work with the outgoing President, the Executive, and the Office Staff to learn the role of President;
	\item be considered an ex-officio non-voting member of Council, the Executive, and all of the committees of the Society
	\item  be compensated at the rate of a Vice-President
	 

\end{longenum}

\end{longenum}

\subsection{Campaign Impropriety and Appeals}

\begin{longenum}[ label*=\thesubsection.\arabic*., align=left]
	\item All members of the Society have the right to submit a charge of campaign impropriety.
		\index{Election!Campaign Impropriety!Allegation}
		\index{Referendum!Campaign Impropriety!Allegation}
		\index{Chief Returning Officer!Campaign Impropreity!Investigation}
    \item Except in those instances outlined in the Society Elections Policy (under Campaign Impropriety and Appeals), where the CRO has determined that an act of campaign impropriety  has occurred, the CRO has the discretion to take the following actions:
   
    \begin{longenum}[ label*=\arabic*., align=left]
		\item Reduce or eliminate a candidate's/referendum group's subsidy;
			\index{Election!Campaign Improriety!Subsidy}
			\index{Referendum!Campaign Impropriety!Subsidy}
					\index{Chief Returning Officer!Campaign Impropreity!Subsidy}
        \item Disqualify the candidate;
\index{Election!Campaign Improriety!Disqualification}
\index{Referendum!Campaign Impropriety!Disqualification}
		\index{Chief Returning Officer!Campaign Impropreity!Disqualification}
        \item Declare the election to be void.
\index{Election!Campaign Improriety!Void Election}
\index{Referendum!Campaign Impropriety!Void Election}
		\index{Chief Returning Officer!Campaign Impropreity!Void Election}
	\end{longenum}
    \item In the event a winning candidate is disqualified, the runner-up will take the place of the disqualified winner.
    \begin{longenum}[ label*=\arabic*., align=left]
		\item In the event that the election is declared void, the election process proceeds as if there were no candidates during the initial candidate nomination period under the Society Elections Policy 4.2.5. 
	\end{longenum}
    \item Appeals regarding sanctions or disqualifications levied by the Chief Returning Officer shall be made in writing to the Appeals Review Commission, care of the Society's office, within forty-eight (48) hours of the announcement. The Appeals Review Commission shall render a decision and make such decision public within two (2) days following their meeting.
    \item The results of the Society's elections/referenda, as accumulated by the on-line polling station shall not be deleted until the deadline for election appeals has passed.
\end{longenum}

\subsection{Validity}

\begin{longenum}[ label*=\thesubsection.\arabic*., align=left]
	\item Any full, associate, or leave-of-absence member of the Society may challenge the validity of an election/referendum in a written submission to the CRO within three (3) business days after the announcement of results. Such submission shall contain the appellant's name, student number, telephone number and UWO email address, as well as a detailed account of the alleged reasons for invalidating the election/referendum.
			\index{Election!Campaign Impropriety!Challenge of Validity}
			\index{Referendum!Campaign Impropriety!Challege of Validity}
			\index{Chief Returning Officer!Campaign Impropreity!Challege of Validity}
	
\end{longenum}

\subsection{Campaign Expenses and Subsidy}

\begin{longenum}[ label*=\thesubsection.\arabic*., align=left]
	\item The Society shall provide reimbursement for campaign expenses incurred by
presidential and referendum campaigns up to 50\% of the maximum spending limit, and 100\% of expenses incurred by Graduate Student Representative to the Board of Governor and Graduate Student Representative to the Senate campaigns.
				\index{Election!Campaign Subsidy}
				\index{Referendum!Campaign Subsidy}
				\index{Chief Returning Officer!Campaign Subsidy}
				\index{Election!Campaign Subsidy}
    \item The CRO shall have the authority to disallow any campaign expenditure. 
\end{longenum}