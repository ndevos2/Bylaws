\begin{center}
	\rule{\textwidth}{3.6pt}\\[\baselineskip] % Thick horizontal linee
	\begin{Huge}
		\textbf{Speaker's Ruling}
	\end{Huge}
	
	\rule{\textwidth}{3.6pt}\\[\baselineskip] % Thick horizontal line
	
	\vspace*{1\baselineskip} % Whitespace Between Title and Discriptive Title
\end{center}
\section*{Motion ruled out of order before January Council, 2022}

\index{Speaker!Rulings}
\noindent On 18 January 2022 at 5:32pm, the SPEC committee\footnote{Deputy Speaker's Note: The SOGS Pilot Project for Executive Compensation (SPEC) was an \emph{ad-hoc} committee that existed from January 2020 through to February 2022} sent the following motion to me to include in the January 2022 Council Package:
\begin{small}
\begin{center}
\noindent\begin{minipage}{0.8\linewidth}
	\emph{Whereas} the current stipend payment for Executive compensation is unclear as it is tied to hourly TA pay; \\
	\emph{Whereas} the Society does not compensate any Executive or Non-Executive officers by an hourly rate; \\
	\emph{BIRT} the Society migrate all volunteer positions to an annual stipend or honorarium separate from the TA Collective Agreement; \\
	\emph{BIFRT} Executive compensation be \$20,000 annually; \\
	\emph{BIFRT} Policy 4.6.6. be changed from ``Non-Executive Compensation Chart'' to ``Volunteer Compensation Chart''; \\
	\emph{BIFRT} Policy 4.6.6. be updated to include ``President'' at \$20,000 annually and ``Vice-Presidents'' at \$20,000 annually; \\
	\emph{BIFRT} Policy 4.6.2. ``Executive members of the Society shall be compensated in proportion to TA wages, as follows: 1.0 TA for Vice-President and 1.5 TA for the President.'' be deleted; \\
	\emph{BIFRT} the migration of remaining non-executive positions is completed by the end of the 2024-2025 fiscal year.
\end{minipage}
\end{center}
\end{small}

\begin{multicols}{2}
\subsection*{Ruling Request}
\noindent The next day, on 19 January 2022 at 1:49pm, the [Vice-President Student Services] sent me a request to make an official ruling regarding whether this motion is out of order or not.\\
\noindent Here is a summary of the [Vice-President Student Services'] reasoning for believing that the SPEC committee’s motion is out of order.\\

In the minutes for the SPEC committee from 15 November 2021 the committee chair presented the following motion: \begin{small}\begin{center}\noindent\begin{minipage}{0.9\linewidth}
				\emph{BIRT} SPEC recommends a \$24,000 flat rate instead of 1.0 TAship for executive positions.
			\end{minipage}
		\end{center}
		\end{small}
The SPEC committee’s minutes for this day show that 1 member of the committee voted in favour of the motion, 2 voted against it, and 1 voted to abstain from the vote. As a result, the motion failed to pass. The SPEC committee has only five voting members, which means that the motion would have required at least 3 votes in favour to pass [\emph{sic}]. \newline
In the [Vice-President Student Services's] reasoning, they note that ``The motion was debated at length by voting members and, ultimately, the committee decided that they did not want to recommend a flat rate for executive compensation,'' and also that ``most of the non-voting members in the room spoke against it as well'' because ``it was not the monetary rate of pay that the committee voted against, but rather the idea of a flat rate.''\\
When the SPEC committee met this month on January 18th the committee voted on the following motion that I will reiterate again here:
	\begin{small}
		\begin{center}
			\noindent\begin{minipage}{0.8\linewidth}
				\emph{Whereas} the current stipend payment for Executive compensation is unclear as it is tied to hourly TA pay;\\
				\emph{Whereas} the Society does not compensate any Executive or Non-Executive officers by an hourly rate;\\
				\emph{BIRT} the Society migrate all volunteer positions to an annual stipend or honorarium separate from the TA Collective Agreement;\\
				\emph{BIFRT} Executive compensation be \$20,000 annually;
				\emph{BIFRT} Policy 4.6.6. be changed from ``Non-Executive Compensation Chart'' to ``Volunteer Compensation Chart'';\\
				\emph{BIFRT} Policy 4.6.6. be updated to include ``President'' at \$20,000 annually and ``Vice-Presidents'' at \$20,000 annually;\\
				\emph{BIFRT} Policy 4.6.2. ``Executive members of the Society shall be compensated in proportion to TA wages, as follows: 1.0 TA for Vice-President and 1.5 TA for the President.'' be deleted;\\
				\emph{BIFRT} the migration of remaining non-executive positions is completed by the end of the 2024-2025 fiscal year.
			\end{minipage}
		\end{center}
	\end{small}

The [Vice-President] Student Services' reasoning in arguing that this motion is out of order is twofold. First, ``Per Robert's Rules, if a motion has been defeated by a majority vote, then a similar motion, which holds the exact same spirit as the defeated motion, cannot be proposed. These two motions \textemdash in November and January \textemdash are both about making executive compensation a flat rate. The defeated November motion proposed a \$24,000 flat rate, whereas this new January motion proposes a \$20,000 flat rate.'' The [Vice-President Student Services] also cites Robert's Rules in their reasoning to note that ``we can make a 'motion to reconsider' a defeated motion BUT [\emph{sic}] it must be made on the same day as the original motion was made. If a motion fails in committee or on our Council floor, it is dead.''

\indent I requested some elaboration in a follow-up email to confirm that the [Vice-President Student Services] was making reference to chapter 37 of Robert's Rules\footnote{Deputy Speaker's Note: The current edition is \emph{RONR (12th ed.)}} on the matter, and they confirmed that that section provided some guidance but that the section in chapter 6 called ``Motions that bring a question again before the assembly'' also provides guidance on the matter. The [Vice-President Student Services] noted that based on what Robert's Rules says in chapter 6, that ``bringing a question again before the assembly means that either it is (1) a completed question during the same session (and thus using the 'reconsider motion'), (2) a motion that was temporarily disposed of (aka, 'take from the table') or (3) a motion previously adopted (aka, a motion that passed).''

In their second email, the [Vice-President Student Services] also noted that the motion is out of order because it goes against Bylaw 2.12.6 in SOGS' unified documents, which reads as follows:
\begin{small}
	\begin{center}
		\begin{minipage}{0.8\linewidth}
			``Proposals for amendments to the budget shall be received by the Vice-President Finance and shall be referred to the Finance Committee. The Finance Committee shall present the proposals with the Finance Committee's recommendation to Council within eight weeks of the Vice-President Finance's receipt of the proposals.''
		\end{minipage}
	\end{center}
\end{small}
In summary, then, the [Vice-President Student Services] is making a twofold argument against the SPEC committee's motion, arguing that it is out of order because it violates the rules in Robert's Rules of Order regarding bringing a failed question again before a committee or assembly and section 2.12.6 of SOGS' unified documents [\emph{sic}].

\subsection*{Out of Order Ruling}
After consulting with our new SOGS Governance staff member, [...\footnote{Deputy Speaker's Note: Removed employee's name}], conducting a minor investigation with SPEC Committee members, and reviewing Robert's Rules of Order and SOGS' unified documents [\emph{sic}], on 20 January 2022 I ruled the SPEC Committee's motion out of order and rejected it from inclusion in the SOGS January Council package.
In rejecting this motion from being included in the Council package I feel that it is my obligation as Speaker to present my reasoning to both the [Vice-President Student Services], who made the request for a ruling, and the SPEC committee, whose motion was ruled out of order.
Regarding the [Vice-President Student Services'] reasoning based on Robert's Rules of Order's guidance on presenting a motion of the same spirit after it has failed, I rule that this motion is in order. Although sections 6:25–6:28 and 37:10 of RR's [\emph{sic}] note that a failed motion cannot be brought again before the assembly except within the same session and by a member of the prevailing party, in this case the matter took place in a committee meeting, which means that section 37:35 of Robert's Rules of Order, ``Reconsideration in Standing and Special Committee'' overrules the aforementioned sections. Section 37:35 notes that
``(1) A motion to reconsider a vote in the committee can be made and taken up regardless of the time that has elapsed since the vote was taken, and there is no time limit to the number of times a question can be reconsidered. Likewise, the rule requiring unanimous consent to renew a defeated motion to Reconsider does not apply in committees.''
It also notes that ``(2) The motion can be made by any member of the committee who did not vote with the losing side; or, in other words, the maker of the motion to Reconsider can be one who voted with the prevailing side, or one who did not vote at all, or even was absent.''
In my investigation of this matter, I emailed members of the SPEC committee regarding their reconsideration of the motion at their 14 January 2022 meeting. The members of the SPEC committee I questioned who voted in both instances noted that they were willing to reconsider the motion because they felt that the spirit of the original motion had changed because the term ``flat rate'' was changed to ``stipend.'' Although a minor change, these members felt that the term flat rate made it sound like the Executive members were being compensated as employees, whereas ``stipend'' made it a different motion because it now reflected that these members are volunteers for the Society. My interpretation of these clauses from Robert's Rules of Order and the reasoning I was provided by SPEC members is that the committee's reconsideration of the motion from the 15 November 2021 meeting at the 18 January 2022 meeting was in order.
However, it is the fact that the motion violates Bylaw 2.12.6. of the Society's unified documents that forced me to rule the motion out of order and to reject it from being added to the agenda this month. In notifying the SPEC committee that I had ruled their motion out of order and rejected it from being added to the agenda, I notified the chair that the committee had the choice to still send the motion to the VP Finance so they could refer it to the Finance Committee.
	
	
\end{multicols}


\vskip 2cm
\noindent
Mitchell Glover \newline
\indent
Speaker of the House \newline
\indent
Society of Graduate Students\newline 
\indent
21 January 2022