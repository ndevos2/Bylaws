\documentclass[12pt,letterpaper,oneside]{book}
\usepackage[left=1.00in, right=1.00in, top=1.00in, bottom=1.00in]{geometry}
\usepackage{makeidx}
\usepackage[english]{babel}
\usepackage[colorinlistoftodos]{todonotes} % Erase once To-Dos are done
\usepackage{draftwatermark} % For committee use when reviewing un-ratified versions of the Unified Documents
\usepackage{easyReview}
\usepackage{multicol}
\usepackage{indentfirst}
\hyphenation{com-mu-ni-cate  ad-mi-nis-tra-tion or-gan-iza-tion ref-er-en-dum U-ni-ver-si-ty}
\usepackage{graphicx}
\setcounter{secnumdepth}{5}
\usepackage{titlesec}
\usepackage{enumitem}
\usepackage{endnotes}
\newlist{longenum}{enumerate}{20}
\setlist[longenum,1]{label=\thesection.\arabic*.}
\setlist[longenum,2]{label=\arabic*.}
\setlist[longenum,3]{label=\arabic*.}
\setlist[longenum,4]{label=\arabic*.}
\setlist[longenum,5]{label=\arabic*.}
\setlist[longenum,6]{label=\arabic*.}
\setlist[longenum,7]{label=\arabic*.}
\setlist[longenum,8]{label=\arabic*.}
\setlist[longenum,9]{label=\arabic*.}
\setlist[longenum,10]{label=\arabic*.}
\setlist[longenum,11]{label=\arabic*.}
\setlist[longenum,12]{label=\arabic*.}
\setlist[longenum,13]{label=\arabic*.}
\setlist[longenum,14]{label=\arabic*.}
\setlist[longenum,15]{label=\arabic*.}
\setlist[longenum,16]{label=\arabic*.}
\setlist[longenum,17]{label=\arabic*.}
\setlist[longenum,18]{label=\arabic*.}
\setlist[longenum,19]{label=\arabic*.}
\setlist[longenum,20]{label=\arabic*.}
\usepackage{lmodern}
\renewcommand*\familydefault{\sfdefault} %% Only if the base font of the document is to be sans serif
\usepackage[T1]{fontenc}
\usepackage[none]{hyphenat}
\title{Bylaws and Constitution}
\makeindex

%----------------------------------------------------------------------------------------
%	TITLE PAGE
%----------------------------------------------------------------------------------------
\newcommand*{\titleGP}{\begingroup % Create the command for including the title page in the document
\centering % Center all text
\vspace*{\baselineskip} % White space at the top of the page

\rule{\textwidth}{1.6pt}\vspace*{-\baselineskip}\vspace*{2pt} % Thick horizontal line
\rule{\textwidth}{0.4pt}\\[\baselineskip] % Thin horizontal line

{\LARGE SOGS ELECTIONS HANDBOOK}

\rule{\textwidth}{0.4pt}\vspace*{-\baselineskip}\vspace{3.2pt} % Thin horizontal line
\rule{\textwidth}{1.6pt}\\[\baselineskip] % Thick horizontal line



\vspace*{2\baselineskip} % Whitespace between location/year and editors

As Compiled on \today \\[\baselineskip]

By the Deputy Speaker

{\Large Nicole Devos \par} % Editor list
\vspace*{\baselineskip} % White space at the top of the page

Under the Supervision of the Speaker \\[\baselineskip]
{\large Kesavi Kanagasabai\par}

\vspace*{2\baselineskip} 
THIS IS A DRAFT COPY AND NOT A RATIFIED VERSION OF THE UNIFIED DOCUMENTS. NOT FOR PUBLIC DISSEMINATION OR USE OUTSIDE OF THE BYLAWS AND CONSTITUTION COMMITTEE.

\vfill % Whitespace between editor names and publisher logo

\includegraphics{logo.jpg}\\[1cm]


{\large THE SOCIETY OF GRADUATE STUDENTS}\par % Publisher
\scshape % Small caps
{\itshape The University Western Ontario\par} % Editor affiliation
London Ontario Canada\par % Location and year
\scshape \the\year \\[0.3\baselineskip] % Year published
\endgroup}
%----------------------------------------------------------------------------------------
%	BLANK DOCUMENT
%----------------------------------------------------------------------------------------

\begin{document} 

\pagestyle{empty} % Removes page numbers 

\titleGP % This command includes the title page

\tableofcontents % This command creates the Table of Contents
\pagebreak % This Command creates a pagebreak at the end of the ToC
\pagestyle{plain}


%%%%These commands are used to insert the Term Limits section from the constitution into this document at the proper numbering
\chapter{Constitution}
\setcounter{section}{6}
\setcounter{subsection}{2}
\setcounter{subsubsection}{6}
\begin{longenum}[ label*=\thesubsubsection.\arabic*., align=left]

\include{Elections/Term_limits}
\end{longenum}

\chapter{Bylaws}
\setcounter{section}{3}
\section{Elections}
\setcounter{subsection}{0}
\setcounter{subsubsection}{0}
\subsection{Responsibility}
\index{Election!Responsibilities of the CRO}
\begin{longenum}[ label*=\thesubsection.\arabic*., align=left]
	\item The Chief Returning Officer (CRO) in conjunction with the Deputy Chief Returning  officer, as per bylaw 2.3, shall be responsible for all aspects of an election or referendum, including but not limited to:
    \begin{longenum}[ label*=\arabic*., align=left]
		\item call for nominations; \index{Election!Call of Nomination}
        \item announcement of candidates;
        \item establishment of an on-line balloting system and the proper functioning of that system; \index{Election!Balloting System}
        \item announcement of the dates of balloting; \index{Election!Date of Election}
        \item ruling on the validity of any election or referendum;
        \item and announcement of official results to candidates, referenda groups, Council, and the media. \index{Election!Anouncement of Result}
	\end{longenum}
    
\end{longenum}

\subsection{Appeals Review Commission}
\index{Appeals Review Commission!Mandate}
\begin{longenum}[ label*=\thesubsection.\arabic*., align=left]
	\item The Appeals Review Commission shall rule on appeals relating to presidential elections and referenda.
    \item The Appeals Review Commission must abide by the following when making rulings: 
    \begin{longenum}[ label*=\arabic*., align=left]
    	\index{Appeals Review Commission!Rules of Procedure}
		\item The onus is on the appellant to demonstrate the charge of impropriety was not correct;
        \item The appellant has the right, but not the obligation, to appear before the Appeals Review Commission at a hearing to present argument or evidence; 
        \item The Appeals Review Commission has the right to initiate and hold an Appeals Review Commission Hearing for an appeal under consideration; 
        \item An Appeals Review Commission Hearing shall request the appearance of individuals relevant to matters of an appeal:
        \begin{longenum}[ label*=\arabic*., align=left]
        	\index{Appeals Review Commission!Summons}
			\item The Appeals Review Commission has the right to demand the presence of the Chief Returning Officer and/or the Deputy Chief Returning Officer at an Election Appeal Hearing; 
            \item The Appeals Review Commission may request, but not demand, the presence of any person not mentioned in Bylaw 2.4.2.2.4.1 at an Election Appeal Hearing;
            \item The procedure and handling of evidence at an Appeals Review Commission Hearing is governed by these bylaws and policies as well as Roberts Rules of Order.
            
		\end{longenum}

	\end{longenum}
    
    \item Members shall be elected to the Appeals Review Commission at the Society's November Council Meeting.
    \index{Election!Appeals Review Commission}
    \index{Appeals Review Commission!Election}
    \item Quorum for a meeting of the Appeals Review Commission shall be four (4).
    \index{Appeals Review Commission!Quorum}
    \index{Quorum!Appeals Review Commission}
\end{longenum}

\subsection{Referendum-Specific Rules}
\index{Referendum}
\begin{longenum}[ label*=\thesubsection.\arabic*., align=left]
	\item Need for a referendum, as well as the wording of the referendum statement, shall be determined and approved by either a General Meeting of the Society or Council. 
    \item A campaign spending limit shall be established by Council for each referendum group.
    \index{Referendum!Spending Limit}
    \index{Election!Spending Limit}
    \item The Chief Returning Officer shall present the decision regarding validity of the referendum to Council for procedural ratification: 
    \begin{longenum}[ label*=\arabic*., align=left]
		\item if valid, the results of the voting shall be binding on the Society;
		\index{Referendum!Binding on Society}
        \item if invalid, the referendum will be re-run at a date determined by Council.
        \index{Referendum!Invalid}
	\end{longenum}
    
\end{longenum}

\subsection{Graduate Student Representative to the Board of Governor and Graduate Student Representative to the Senate Election Rules}
\index{Board of Governor!Election}
\index{Senate!Election}
\index{Election!Senate}
\index{Election!Board of Governor}
\begin{longenum}[ label*=\thesubsection.\arabic*., align=left]
 
\item Timeline for the elections relating to the Board of Governor and University Senate
shall be set by the University Secretariat
\item  All Graduate Students (as defined by the School of Graduate and Postdoctoral Studies) are eligible to vote in this election
\end{longenum}


\subsection{Presidential Election-specific Rules}
\index{President, The!Election}
\index{Election!President}
\begin{longenum}[ label*=\thesubsection.\arabic*., align=left]
	\item Timeline for the Presidential Election shall be set by the University Secretariat.
    \item All full, associate, and leave-of-absence members shall be eligible to vote. \index{Election!Suffrage}
\end{longenum}

\subsection{Campaigning}
\index{Election!Campaign}
\index{Referendum!Campaign}
\begin{longenum}[ label*=\thesubsection.\arabic*., align=left]
    \item Prior to the beginning of the campaign period, all parties and their campaign
managers shall attend a mandatory all candidates meeting wherein they will be briefed on the terms of the election by the CRO and DCRO.
    \index{Election!Campaign!All Candidates Meeting}
    \index{Deputy Chief Returning Officer!All Candidates M eeting}
    \index{Referendum!All Candidates Meeting}
    \index{Chief Returning Officer!All Candidates Meeting}
    \item No candidate may address either Council or any other Society-organized meeting of members without the same opportunity being provided to all candidates.
    \index{Election!Campaign!Equal Time}
    \index{Referendum!Campaign!Equal Time}
\end{longenum}
\subsection{Voting}
\begin{longenum}[ label*=\thesubsection.\arabic*., align=left]
	\item A secret ballot shall be held following the close of the campaign period, for the Presidential election, or during a period determined by Council, for a referendum.
	\index{Election!Vote!Ballot}
	\index{Referendum!Vote!Ballot}
	\begin{longenum}[ label*=\arabic*., align=left]
		\item On-line polling for the Presidential election shall be conducted by the University Secretariat.
		\item In the event that a second nomination period is required for the Presidential election (see 3.3.2.5.) and nominations are received by the CRO during that second nomination period, the following protocols shall be followed:
		\begin{longenum}[ label*=\arabic*., align=left]
			\item On-line polling shall be conducted by the CRO, with assistance from the DCRO;
			\item The polling period shall be determined by the CRO, in accordance with 2.4.7.2. The polling period shall take place no later than the third full week of March;
			\item No access shall be granted to the results of on-line polling during the election, by any member or employee of the Society, with the exception of the CRO and DCRO. The CRO and DCRO must verify daily  that the on-line polling station remains operational, with the CRO taking ultimate responsibility.
			\end{longenum}
		 	\end{longenum}
    \item The regular polling period shall consist of thirty-six (36) consecutive hours at the end of the campaign period.
    	\index{Election!Vote!Voting Period}
    	\index{Referendum!Vote!Voting Period}
   
\end{longenum}
\subsection{Counting Ballots}

\begin{longenum}[ label*=\thesubsection.\arabic*., align=left]
	\item The Presidential election and all referenda shall have no quorum unless otherwise mandated by council.
		\index{Election!Quorum}
		\index{Referendum!Quorum}
		\index{Quorum!Presidential Election}
		\index{Quorum!Refrendum}
    \item A plurality of ballots cast will determine the result of any referendum or election.
    \item In the event of a tie, the tie shall be broken by a vote by Council.
    	\index{Election!Vote!Tie}
    	\index{Referendum!Vote!Tie}
\end{longenum}

\subsection{Transition}
	\index{President!Transition}
	\index{President!President-Elect}
	\index{President-Elect}

\begin{longenum}[ label*=\thesubsection.\arabic*., align=left]
	\item For the month prior to taking office, the President-Elect shall:
	 \begin{longenum}[ label*=\thesubsection.\arabic*., align=left]
	\item Be expected to work with the outgoing President, the Executive, and the Office Staff to learn the role of President;
	\item be considered an ex-officio non-voting member of Council, the Executive, and all of the committees of the Society
	\item  be compensated at the rate of a Vice-President
	 

\end{longenum}

\end{longenum}

\subsection{Campaign Impropriety and Appeals}

\begin{longenum}[ label*=\thesubsection.\arabic*., align=left]
	\item All members of the Society have the right to submit a charge of campaign impropriety.
		\index{Election!Campaign Impropriety!Allegation}
		\index{Referendum!Campaign Impropriety!Allegation}
		\index{Chief Returning Officer!Campaign Impropreity!Investigation}
    \item Except in those instances outlined in the Society Elections Policy (under Campaign Impropriety and Appeals), where the CRO has determined that an act of campaign impropriety  has occurred, the CRO has the discretion to take the following actions:
   
    \begin{longenum}[ label*=\arabic*., align=left]
		\item Reduce or eliminate a candidate's/referendum group's subsidy;
			\index{Election!Campaign Improriety!Subsidy}
			\index{Referendum!Campaign Impropriety!Subsidy}
					\index{Chief Returning Officer!Campaign Impropreity!Subsidy}
        \item Disqualify the candidate;
\index{Election!Campaign Improriety!Disqualification}
\index{Referendum!Campaign Impropriety!Disqualification}
		\index{Chief Returning Officer!Campaign Impropreity!Disqualification}
        \item Declare the election to be void.
\index{Election!Campaign Improriety!Void Election}
\index{Referendum!Campaign Impropriety!Void Election}
		\index{Chief Returning Officer!Campaign Impropreity!Void Election}
	\end{longenum}
    \item In the event a winning candidate is disqualified, the runner-up will take the place of the disqualified winner.
    \begin{longenum}[ label*=\arabic*., align=left]
		\item In the event that the election is declared void, the election process proceeds as if there were no candidates during the initial candidate nomination period under the Society Elections Policy 4.2.5. 
	\end{longenum}
    \item Appeals regarding sanctions or disqualifications levied by the Chief Returning Officer shall be made in writing to the Appeals Review Commission, care of the Society's office, within forty-eight (48) hours of the announcement. The Appeals Review Commission shall render a decision and make such decision public within two (2) days following their meeting.
    \item The results of the Society's elections/referenda, as accumulated by the on-line polling station shall not be deleted until the deadline for election appeals has passed.
\end{longenum}

\subsection{Validity}

\begin{longenum}[ label*=\thesubsection.\arabic*., align=left]
	\item Any full, associate, or leave-of-absence member of the Society may challenge the validity of an election/referendum in a written submission to the CRO within three (3) business days after the announcement of results. Such submission shall contain the appellant's name, student number, telephone number and UWO email address, as well as a detailed account of the alleged reasons for invalidating the election/referendum.
			\index{Election!Campaign Impropriety!Challenge of Validity}
			\index{Referendum!Campaign Impropriety!Challege of Validity}
			\index{Chief Returning Officer!Campaign Impropreity!Challege of Validity}
	
\end{longenum}

\subsection{Campaign Expenses and Subsidy}

\begin{longenum}[ label*=\thesubsection.\arabic*., align=left]
	\item The Society shall provide reimbursement for campaign expenses incurred by
presidential and referendum campaigns up to 50\% of the maximum spending limit, and 100\% of expenses incurred by Graduate Student Representative to the Board of Governor and Graduate Student Representative to the Senate campaigns.
				\index{Election!Campaign Subsidy}
				\index{Referendum!Campaign Subsidy}
				\index{Chief Returning Officer!Campaign Subsidy}
				\index{Election!Campaign Subsidy}
    \item The CRO shall have the authority to disallow any campaign expenditure. 
\end{longenum}


\setcounter{section}{7}
\setcounter{subsection}{0}
\setcounter{subsubsection}{0}
\section{Elections Table}
\index{Election!Dates}

\begin{table}[h!]
	\centering
    \begin{tabular}{ l |c   c }

    Position & Month of election & Day of taking office \\  \hline
    Appeals Review Commission & November & December 1\\ 
    President & As established by the & May 1\\
     & University Secretariat & \\
    Vice-President Finance  & April & May 1 \\ 
    Vice-President Student Services & April & May 1 \\ 
    Speaker & June & July 1 \\ 
    Ombudsperson & July & August 1\\
    Vice-President Academic & April & May 1 \\ 
    Vice-President Advocacy & April & May 1 \\ 
    Commissioners & October & November 1 \\
    \add{Sustainability Coordinator}\asttablefootnote & \add{October}\astref & \add{November 1}\astref \\ 
    Chief Returning Officer & November & January 1 \\ 
    Graduate Representative to the Senate &\multicolumn{2}{c}{As established by the University Secretariat}\\
	Graduate Representative to the Board of Governors  &\multicolumn{2}{c}{As established by the University Secretariat}\\	
\end{tabular}
\end{table}


\newpage


\chapter{Policy}
\setcounter{section}{2}
\section{Elections Policy}
\index{Election!Policy}
\begin{longenum}[ label*=\thesection.\arabic*., align=left]
 \item \textbf{Appeals Review Commission}
 \begin{longenum}[label*=\arabic*., align=left]
  \item Appeals Review Commission (ARC) shall be elected annually at the Society's November Council meeting.
  \item Appeals Review Commission (ARC) shall have a ranked waiting list
  \begin{longenum}[label*=\arabic*., align=left]
  	\item  During the November council meeting, a waiting list of no less than one member shall be created.
  	\item Should a member of the Appeals Review Commission no longer be able to fulfil their duty, the highest ranked member from the waiting list shall take their place.
  \end{longenum}
  \item Any member who joins the Appeals Review Commission or Appeals Review Commission waiting list must state their intention to not graduate within four (4) months.
  \end{longenum}

 \item \textbf{Presidential Nominations}
 \begin{longenum}[label*=\arabic*., align=left]
  \item  Nominations for the position of President shall be open from the first Monday following January 2nd for ten (10) consecutive business days, closing on the tenth day at the close of business hours. 
  \item  Nomination forms must be obtained from the Society office during the Society's business hours. Nominations shall bear the signatures of two full, associate, or leave-of-absence members of the Society as well as that of  the nominee, and shall be submitted to the Chief Returning Officer during the Society's business hours.
  \item Appeals regarding a decision on the validity of a nomination shall be made in writing to the Chief Returning Officer no later than one business day after the announcement nomination. The Chief Returning Officer shall render a decision before the beginning of the campaign period.
  \item The call for nominations shall be advertised by the CRO one week prior to the opening of nominations.
  \item In the event that no nominations for the position of President be received by the Chief Returning Officer by the close of nominations, nominations shall be reopened the first Monday in February and close after ten (10) consecutive business days, at the close of business hours.
  \item In the event that no nominations be received by the end of the second nomination period, the Council shall nominate an eligible member of the Society at the Council meeting following the close of nominations. If Council fails to do so, then the Executive shall nominate an eligible member of the Society as President, subject to ratification at the next Council meeting. If ratification is not forthcoming, Council must appoint a President at that meeting. If Council fails to do so, then the new President shall be selected by the Executive from the Vice-Presidents.
  \item The new President shall take the following Oath at the Council meeting following their election: ``I,[name], do hereby swear to faithfully fulfill my duties as President and to uphold the Constitution and Bylaws of the Society of Graduate Students of the University of Western Ontario." The Oath shall be administered by the Speaker.
  \item In the event that the elected President is a departmental or part-time representative, Speaker, or another member of the Executive, that person shall not hold the position of departmental representative, Speaker or other Executive member while being President.
  \item Actual transfer of signing authority and responsibilities of the Office of the President shall take place on the first of May, at which time the retiring President shall formally hand over office to the incoming president. In the event that there is a president to take the office, otherwise it will happen as soon as practicable
   \end{longenum}

 \item \textbf{Campaigning}
  \begin{longenum}[label*=\arabic*., align=left]
 \item Campaigning is defined as any action by a candidate or campaign manager, or any action undertaken at the behest of the candidate or campaign manager, that is intended to influence any voter to cast their ballot on behalf of the candidate in question.
\item The campaign period shall begin two business days after nominations close and shall continue up to and including the day preceding the balloting. No campaigning may be undertaken other than during the designated campaign period.
\item Prior to the beginning of the campaign period, all parties and their campaign managers shall attend a meeting wherein they will be briefed on the terms of the election by the Chief Returning Officer and Deputy Chief Returning Officer.
\item Campaigning must cease by midnight (11:59 pm local time) before the day of balloting. All advertising must be removed by this deadline. 
    \index{Election!Campaign!End of Period}
    \index{Referendum!Campaign!End of Period}
 \item No candidate may address either Council or any other Society-organized meeting of members without the same opportunity being provided to all candidates.
    \index{Election!Campaign!Equal Time}
    \index{Referendum!Campaign!Equal Time}
\item At that meeting an agreement will be signed by all parties wherein they acknowledge being bound to conduct their campaign in accordance with the Society's Constitution and Bylaws, and define any terms of the election not covered by these Bylaws.
\item All  forms of media may be used during  the campaign period. No media coverage is allowed on  the day of balloting.
\item  Society resources or materials may not be used in the preparation of campaign material, except with the agreement of all candidates and the Chief Returning Officer.
\item Society space may not be used in the dissemination of campaign material, except with the agreement of all candidates and the Chief Returning Officer.
 \end{longenum}

 \item 	\textbf{Voting Operations}
\begin{longenum}[label*=\arabic*., align=left]
\item Voting
	\begin{longenum}[label*=\arabic*., align=left]
	\item A secret ballot shall be held following the close of the campaign period, for the Presidential election, or during a period determined by Council, for a referendum.
		\begin{longenum}[label*=\arabic*., align=left]
		\item On-line polling for the Presidential election shall be conducted by the University Secretariat.
		\item In the event that a second nomination period is required for the Presidential election (see 3.3.2.5.) and nominations are received by the CRO during that second nomination period, the following protocols shall be followed:
			\begin{longenum}[label*=\arabic*., align=left]
			\item On-line polling shall be conducted by the CRO, with assistance from the DCRO;
			\item The polling period shall be determined by the CRO, in accordance with 2.4.7.2. The polling period shall take place no later than the third full week of March;
			\item No access shall be granted to the results of on-line polling during the election, by any member or employee of the Society, with the exception of the CRO and DCRO. The CRO and DCRO must verify daily that the on-line polling station remains operational, with the CRO taking ultimate responsibility.
			\end{longenum}
		\end{longenum}
	\item The regular polling period shall consist of thirty-six (36) consecutive hours at the end of the campaign period.
	\end{longenum}		

\item Counting Ballots
	\begin{longenum}[label*=\arabic*., align=left]
	\item The Presidential election and all referenda shall have no quorum unless otherwise mandated by council.
	\item A plurality of ballots cast will determine the result of any referendum or election.
	\item In the event of a tie, the tie shall be broken by a vote by Council.
	\end{longenum}
\end{longenum}


 \item 	\textbf{Advertisements	of	a	Presidential	Election	or	Referendum	by	the	Society }
  \begin{longenum}[label*=\arabic*., align=left]
  \item Advertisements shall be placed:
 \begin{longenum}[label*=\arabic*., align=left]
\item on campus bulletin boards one month prior to the final voting date;
\item on the Society's Web page and in the monthly newsletter;
\item Any other place as the CRO considers appropriate and effective within budgetary 
constraints.
 \end{longenum}
\item Advertisements shall consist of date(s) and place(s) or voting, the candidates' names or referendum 
statement(s) and the name of the Society. 
\end{longenum}

  
 \item \textbf{Scrutineer}
 \index{Election!Scrutineer}
 \begin{longenum}[label*=\arabic*., align=left]
\item Each candidate/referendum group shall be allowed one scrutineer to be present when the final results of the election/referendum, as recorded by the on-line polling station, are revealed.
 \end{longenum}
 \item 	\textbf{Campaign Impropriety and Appeals Policy}	
 \begin{longenum}[label*=\arabic*., align=left]
\item No candidate/referendum group or individual member may, using their own initiative and 
discretion, attempt to enforce the rules for elections.
\item Campaign Impropriety is defined as any action undertaken by a candidate or their representative 
during the campaign and polling period that can be shown to be violations of any part of the Society's 
Constitution, Bylaws, or the agreement between the parties reached at the All Candidates Meeting.
\item To find that an act of campaign impropriety has occurred, the CRO must be satisfied that the result of 
the Deputy CRO's investigation shows, on the balance of probabilities, the indicted party did commit the 
offence with which they have been charged.
\item The investigation
\begin{longenum}[label*=\arabic*., align=left]
\item All charges of campaign impropriety shall be submitted in writing to the Deputy Chief Returning Officer, (Deputy CRO) at the Society office or by email to the official CRO/DCRO email address;
\item Within one business day of the charge having been submitted the Deputy CRO will begin an impartial investigation and will also notify the CRO of the pending investigation; 
\item Barring exceptional circumstances, the Deputy CRO shall submit the result of their investigation to the CRO no more than one business day after the start of investigation.
\end{longenum}
\item Should a candidate/referendum group receive a request from the Deputy CRO to provide factual information which wholly pertains to an investigation of campaign impropriety, the candidate/ referendum group must respond within one half of the investigation time or twenty-four (24) hours, whichever is greater. Failure to respond within the time frame will result in a report of nonresponse in the Deputy CRO's written report.
\item The Chief Returning Officer shall decide on any charge or campaign impropriety, no more than four (4) business days following the complaint.
\item Sanctions
	\begin{longenum}[label*=\arabic*., align=left]
\item It is at the CRO's discretion to determine whether a sanction shall be one of the following three types: Warning, Minor, or Major. The CRO, DCRO, candidates, and their campaign managers shall determine the parameters of what constitutes campaign impropriety at the All Candidates Meeting (see Bylaw 2.4.6.2.).
\item Warning – the CRO shall issue a warning to the candidate via email regarding their campaign impropriety.
\item Minor Sanctions – The CRO may use the office's authority to rectify the situation in the event of a minor sanction (see Bylaw 2.4.10.).
\item Major Sanctions – Violations of the following nature will result in automatic disqualification of the candidate:
	\begin{longenum}[label*=\arabic*., align=left]
\item Failure of the candidate and their campaign manager to attend the All Candidates' Meeting with the Chief Returning Officer and Deputy Chief Returning Officer;
\item Tampering with other candidates' signs so as to cause their being defaced or removed;
\item Spending 125\% or more of the maximum spending limit;
\item Violations of Canadian and Ontario law may result in disqualification at the CRO's discretion.
\end{longenum}
\end{longenum}
 \end{longenum}
 
 \item \textbf{Appeals}
 \index{Election!Appeals}
 \index{Election!Elections Appeals Commission}
 \begin{longenum}[label*=\arabic*., align=left]
 \item The Deputy CRO shall investigate the appeal and present the results of that investigation to the CRO within  two (2) business days. The CRO will  take appropriate action in response  to  the investigation results within two (2) business days.
\item  When  the  Chief  Returning  Officer  has  rendered  a  decision  on  the  submission,  a  further  written appeal may be taken to the Appeals Review Commission within two (2) business days of the announcement of the decision.
\item The onus is on the appellant to appeal the decision of the CRO in writing to necessitate the activation of the Appeals Review Commission.
  \end{longenum}
  \item \textbf{Campaign Expenses and Subsidy Policy}	
  \begin{longenum}[label*=\arabic*., align=left]
  \item The spending limit will be \$150 unless changed at the February Council meeting by a motion which requires a simple majority to pass.
\item \label{Presidential Subsidy}The Society shall provide full subsidies for campaign expenses incurred by each referendum group or presidential candidate that receive at least 10\% of the total unspoiled votes cast in the election.  
			\begin{longenum}[label*=\arabic*., align=left]
			\item  Acclaimed  presidential  candidates  are  assumed  to  have  passed  the  10\%  threshold  in section \ref{Presidential Subsidy}.
			\end{longenum}
\item All candidates shall submit to the Chief Returning Officer documentation of all expenditures by the election day.
 \end{longenum}
\end{longenum}

\printindex

\end{document}
